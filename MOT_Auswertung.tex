% !TEX encoding = UTF-8 Unicode
\documentclass[12pt,a4paper]{article}
\usepackage[ngerman]{babel}
\usepackage[utf8]{inputenc}
\usepackage[T1]{fontenc}
\usepackage{lmodern}
\usepackage{graphicx}
\usepackage{enumerate}
\usepackage[]{epstopdf}
\usepackage[margin=1in]{geometry}
\usepackage{titling}
\usepackage{hyperref}
\usepackage[german]{cleveref}
\usepackage{amsmath,amsthm,verbatim,amssymb,amsfonts,amscd}
\renewcommand{\familydefault}{\sfdefault}
\usepackage[miktex]{gnuplottex}
\usepackage[decimalsymbol=comma,separate-uncertainty=true,expproduct=\cdot,
            uncertainty-separator=\pm]{siunitx}
\setlength{\droptitle}{-2cm}
\setlength{\parindent}{0pt}
\title{Auswertung zu IQ13:\\
       magneto-optische Falle\\
       WiSe 15/16\\
       Block III}
\author{Ramin Javadi 2993630 und Felix Schrader 3053850}
\date{}
\begin{document}
\maketitle
\tableofcontents
\pagebreak
\section{Theorie einer MOT}
  \subsection{Optische Melasse}
  \subsection{Magnetfeld}
  \subsection{Übergänge von ${}^{87}$Rb}
\section{Bestandteile der MOT}
  \subsection{Die Laser}
  \subsection{Der AOM}
  \subsection{Laserkopplung}
  \subsection{Der TA}
  \subsection{Die Vakuumkammer}
  \subsection{Die Kamera}
\section{Messungen}
  \subsection{Intensitätsmessungen im Strahlengang}
  \subsection{Vermessung der Strahlbreite}
  \subsection{Eichung der Kamera}
\end{document}